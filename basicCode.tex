
\subsection{Code Syntax}

\frame{\frametitle{Code Syntax}
    \begin{description}
        \item[\%] Comment
        \item[\textbackslash] Start of a command
        \begin{description}
            \item[\command{LaTeX}] \LaTeX
        \end{description}
        \item[\brackets{}] Used for options on commands
        \item[\{\}] Used for arguments on commands
    \end{description}
}


\subsection{\braceCommand{documentclass}}

\frame{\frametitle{\braceCommand{documentclass}}
    Used to define the type of document\\
    Example types of documentclass:
    \begin{description}
        \item[article] Good for scientific journals, presenations, short reports, documentation, \ldots
        \item[report] Good for long reports, small books, thesis, \ldots
        \item[book] Good for writing real books
        \item[slides] Good for simple slides
        \item[beamer] Good for complex presenations
    \end{description}
}
\frame{\frametitle{\braceCommand{documentclass}}
    Options:
    \begin{description}
        \item[pt] Font size. Default is 10pt. Ex: 12pt, 14pt, \ldots
        \item[letterpaper] Size of the paper. letterpaper is default.
        \item[twocolumn] Define if the document should be one column or two
    \end{description}
}


\subsection{\braceCommand{usepackage}}

\frame{\frametitle{\braceCommand{usepackage}}
    Used to add modules/packages to the \LaTeX\ document.\\
    Example packages:
    \begin{description}
        \item[geometry] Used to change geometry including margin and layout
        \item[graphicx] Used to include graphics in the document
        \item[float] Allows changing the position of a graphic in the document/page
    \end{description}
}
