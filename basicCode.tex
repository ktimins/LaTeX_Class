
\subsection{Code Syntax}

\begin{frame}
\frametitle{Code Syntax}
    \begin{description}
        \item[\%] Comment
        \item[\textbackslash] Start of a command
        \begin{description}
            \item[\command{LaTeX}] \LaTeX
        \end{description}
        \item[\brackets{}] Used for options on commands
        \item[\{\}] Used for arguments on commands
    \end{description}
\end{frame}




\subsection{\braceCommand{documentclass}}

\begin{frame}
\frametitle{\braceCommand{documentclass}}
    Used to define the type of document\\
    Example types of documentclass:
    \begin{description}
        \item[article] Good for scientific journals, presentations, short reports, documentation, \ldots
        \item[report] Good for long reports, small books, thesis, \ldots
        \item[book] Good for writing real books
        \item[slides] Good for simple slides
        \item[beamer] Good for complex presentations
    \end{description}
\end{frame}


\begin{frame}
\frametitle{\braceCommand{documentclass}}
    Options:
    \begin{description}
        \item[pt] Font size. Default is 10pt. Ex: 12pt, 14pt, \ldots
        \item[letterpaper] Size of the paper. letterpaper is default.
        \item[twocolumn] Define if the document should be one column or two
    \end{description}
\end{frame}




\subsection{\braceCommand{usepackage}}

\begin{frame}
\frametitle{\braceCommand{usepackage}}
    Used to add modules/packages to the \LaTeX\ document.\\
    Example packages:
    \begin{description}
        \item[geometry] Used to change geometry including margin and layout
        \item[graphicx] Used to include graphics in the document
        \item[float] Allows changing the position of a graphic in the document/page
    \end{description}
\end{frame}



\subsection{\braceCommand{begin} \&\& \braceCommand{end}}

\begin{frame}
\frametitle{\braceCommand{begin} \&\& \braceCommand{end}}
    Used to start and end \textit{environments} in \LaTeX .\\
    Some environments that use \braceCommand{begin} and 
    \braceCommand{end}:
    \begin{itemize}
        \item title
        \item document
        \item lists
%        \item figure
    \end{itemize}
\end{frame}



\subsection{Lists}

\begin{frame}
\frametitle{Lists}
    \begin{itemize}
        \item Lists come in three flavors:
        \begin{itemize}
            \item itemize
            \item enumerate
            \item description
        \end{itemize}
    \end{itemize}
\end{frame}



\subsubsection{itemize}

\begin{frame}[fragile]
\frametitle{itemize}
\begin{teX}
\begin{itemize}
    \item This is a sample list
    \item to show the code for an itemized list
    \begin{itemize}
        \item sublist
    \end{itemize}
    \item Another item
\end{itemize}
\end{teX}
\end{frame}

\begin{frame}
    \begin{itemize}
        \item This is a sample list
        \item to show the code for an itemized list
        \begin{itemize}
            \item sublist
        \end{itemize}
        \item Another item
   \end{itemize}
\end{frame}

\subsubsection{enumerate}

\begin{frame}[fragile]
\frametitle{enumerate}
\begin{teX}
\begin{enumerate}
    \item This is a sample list
    \item to show the code for an enumerated list
    \begin{enumerate}
        \item sublist
    \end{enumerate}
    \item Another item
\end{enumerate}
\end{teX}
\end{frame}

\begin{frame}
\frametitle{enumerate}
    \begin{enumerate}
        \item This is a sample list
        \item to show the code for an enumerated list
        \begin{enumerate}
            \item sublist
        \end{enumerate}
        \item Another item
   \end{enumerate}
\end{frame}

\subsubsection{description}

\begin{frame}[fragile]
\frametitle{description}
\begin{teX}
\begin{description}
    \item[description] Good for definitions
    \item[something] This shows a second definition
\end{description}
\end{teX}
\end{frame}

\begin{frame}
\frametitle{description}
\begin{description}
    \item[description] Good for definitions
    \item[something] This shows a second definition
\end{description}
\end{frame}


\section{Sectioning}

\begin{frame}
\frametitle{Sectioning}
\begin{description}
    \item[\braceCommand{chapter}] Level 0\\New page\\Only 
    in books and reports
    \item[\braceCommand{section}] Level 1\\Same page
    \item[\braceCommand{subsection}] Level 2\\Same page
    \item[\braceCommand{subsubsection}] Level 3\\Same page
\end{description}
\end{frame}


\section{Math}

\subsection{Math}

\begin{frame}
\frametitle{Math}
\begin{itemize}
    \item Math in \LaTeX{} is beautiful.
    \item Uses the asamath or mathtools package
    \item Can be sectioned off or inline
    \item Relation Symbols, Binary Operations, Greek Letters, Set/Logic
        Notation, Delimiters, Trig Symbols, etc.
    \item Simple example: \$2+2=5\$ $\implies 2+2=5$
\end{itemize}
\end{frame}

\subsection{Ex.}

\begin{frame}
\frametitle{Math Ex.}
\begin{math}
A_{m,n} =
 \begin{pmatrix}
  a_{1,1} & a_{1,2} & \cdots & a_{1,n} \\
  a_{2,1} & a_{2,2} & \cdots & a_{2,n} \\
  \vdots  & \vdots  & \ddots & \vdots  \\
  a_{m,1} & a_{m,2} & \cdots & a_{m,n}
 \end{pmatrix}
 \end{math}
\end{frame}


\section{Document Ex.}

\begin{frame}
\frametitle{Document Ex.}
\begin{figure}[DocEx] \centering{
\includegraphics[trim=0cm 2cm 0cm 6cm, crop=true, 
                 height=1.3\textheight]{docEx.pdf}}
\caption{Example Document}
\end{figure}
\end{frame}
