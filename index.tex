\documentclass[compress,14pt]{beamer}
\usepackage{etex}
\mode<presentation>

\usetheme{Warsaw}
\setbeamercolor*{palette primary}{use=structore, fg=white, bg=blue}
\defbeamertemplate*{footline}{shadow theme}
{%
  \leavevmode%
  \hbox{\begin{beamercolorbox}[wd=.5\paperwidth,ht=2.5ex,dp=1.125ex,leftskip=.3cm plus1fil,rightskip=.3cm]{author in head/foot}%
    \usebeamerfont{author in head/foot}\hfill\insertshortauthor%
  \end{beamercolorbox}%
  \begin{beamercolorbox}[wd=.5\paperwidth,ht=2.5ex,dp=1.125ex,leftskip=.3cm,rightskip=.3cm plus1fil]{title in head/foot}%
      \usebeamerfont{title in head/foot}\insertshorttitle\hfill\insertframenumber\,/\,\inserttotalframenumber%
  \end{beamercolorbox}}%
  \vskip0pt%
}
\setbeamertemplate{navigation symbols}{}%remove navigation symbols

\usepackage{pdfpages}
\usepackage{subfigure}
\usepackage{multicol}
\usepackage{mathtools}
\usepackage{epsfig}
\usepackage{graphicx}
\usepackage[all,knot]{xy}
\xyoption{arc}
\usepackage{url}
%\usepackage[hypertexnames=false, bookmarks=true, bookmarksnumbered=true, breaklinks=true, linkbordercolor={0 0 1}]{hyperref}
\usepackage{multimedia}
\usepackage[T1]{fontenc}
\usepackage{bera}
\usepackage{listings}
\usepackage{verbatim}

%% Define a new 'leo' style for the package that will use a smaller font.
\makeatletter
\def\url@leostyle{%
  \@ifundefined{selectfont}{\def\UrlFont{\sf}}{\def\UrlFont{\small\ttfamily}}}
\makeatother
%% Now actually use the newly defined style.
\urlstyle{leo}

\lstnewenvironment{teX}[1][]
{\lstset{language=[LaTeX]TeX}\lstset{escapeinside={(*@}{@*)},
    numbers=left,numberstyle=\normalsize,stepnumber=1,numbersep=5pt,
    breaklines=true,
    %firstnumber=last,
    %frame=tblr,
    framesep=5pt,
    basicstyle=\normalsize\ttfamily,
    showstringspaces=false,
    keywordstyle=\itshape\color{blue},
    %identifierstyle=\ttfamily,
    stringstyle=\color{maroon},
    commentstyle=\color{black},
    rulecolor=\color{black},
    xleftmargin=0pt,
    xrightmargin=0pt,
    aboveskip=\medskipamount,
    belowskip=\medskipamount,
    backgroundcolor=\color{white}, #1
                                    }}
{}

% define your own colours:
\definecolor{innotek}{RGB}{1,54,160}
\definecolor{Red}{rgb}{1,0,0}
\definecolor{Blue}{rgb}{0,0,1}
\definecolor{Green}{rgb}{0,1,0}
\definecolor{magenta}{rgb}{1,0,.6}
\definecolor{lightblue}{rgb}{0,.5,1}
\definecolor{lightpurple}{rgb}{.6,.4,1}
\definecolor{gold}{rgb}{.6,.5,0}
\definecolor{orange}{rgb}{1,0.4,0}
\definecolor{hotpink}{rgb}{1,0,0.5}
\definecolor{newcolor2}{rgb}{.5,.3,.5}
\definecolor{newcolor}{rgb}{0,.3,1}
\definecolor{newcolor3}{rgb}{1,0,.35}
\definecolor{darkgreen1}{rgb}{0, .35, 0}
\definecolor{darkgreen}{rgb}{0, .6, 0}
\definecolor{darkred}{rgb}{.75,0,0}

\xdefinecolor{olive}{cmyk}{0.64,0,0.95,0.4}
\xdefinecolor{purpleish}{cmyk}{0.75,0.75,0,0}

\useoutertheme[subsection=false]{smoothbars}

\def\brackets#1{[#1]}
\def\braceCommand#1{\textbackslash #1\{\}}
\def\command#1{\textbackslash #1}
\def\tab{\hspace{1em}}
\AtBeginSection[]{
    \begin{frame}{Outline}
        \begin{multicols}{2}
            \tableofcontents[currentsection,
                             subsectionstyle=show/show/hide,
                             subsubsectionstyle=hide/hide]
        \end{multicols}
    \end{frame} }

\title{Intro to \LaTeX{}}
\subtitle{Learn to Write Basic \LaTeX{} Documents}
\author{Kyle Timins}
\institute[Vermont Technical College]{Vermont Technical College}
\date{\today{}}

\begin{document}

\begin{frame}
    \titlepage
\end{frame}

\section[Outline]{}
\begin{frame}{Outline}
    \begin{multicols}{2}
        \tableofcontents[hideallsubsections]
    \end{multicols}
\end{frame}

\section{Overview}
\subsection{What is \LaTeX}

\begin{frame}[allowframebreaks=0.8]
\frametitle{What is \LaTeX}
    \begin{itemize}
        \item \LaTeX{} --- Pronounced 'Lay-Tech'
        \item 'Tis not a fetish
        \item Document markup language
        \begin{itemize}
            \item Similar to Markdown or HTML
        \end{itemize}
        \item Document preparation system
        \framebreak
        \item Types of files you will/could write
        \begin{description}
          \item[.tex] Contains the actual document
          \item[.cls] Class files containing settings for you document 
          \item[.sty] LaTeX Macro package. Load with 
                      \textbackslash usepackage.
          \item[.bib] The data base file containing your bibliography 
                      entries.
        \end{description}
    \end{itemize}
\end{frame}

\subsection{Why Use \LaTeX{}}

\begin{frame}
\frametitle{Why Use \LaTeX{}}
    \begin{itemize}
        \item Both \LaTeX\ and \TeX\ are stable
        \item \LaTeX\ is free
        \item Available on most operating systems
    \end{itemize}
\end{frame}



\subsection{Pros}

\begin{frame}[allowframebreaks=0.8]
\frametitle{Pros}
    \begin{itemize}
        \item Available on most operating system
        \item Uses bibtex for citations
        \begin{itemize}
            \item Does formatting for you based on given details
        \end{itemize}
        \framebreak
        \item Uses text files rather than binary
        \begin{itemize}
            \item Can be used on source management systems
            \begin{itemize}
                \item GIT, SVN, HG, ...
            \end{itemize}
            \item Can be developed by multiple people
            \begin{itemize}
                \item Work on different text files
                \item Compile into one document
            \end{itemize}
            \item Can be created on any text editor
            \begin{itemize}
                \item No restrictions on any programs
                \item Allows easy transfer of files
            \end{itemize}
            \item Can be created with multiple files included in a main file
        \end{itemize}
        \framebreak
        \item Can be created and compiled online
        \item Math equations is beautiful
        \item Great for long documents like dissertations
        \item Not a WYSIWYG
    \end{itemize}
\end{frame}



\subsection{Cons}

\begin{frame}
\frametitle{Cons}
    \begin{itemize}
        \item Including graphics/figures can be finicky
        \begin{itemize}
            \item But can be persuaded
        \end{itemize}
        \item Not a WYSIWYG
    \end{itemize}
\end{frame}





\section{Basic Code}

\subsection{Code Syntax}

\begin{frame}
\frametitle{Code Syntax}
    \begin{description}
        \item[\%] Comment
        \item[\textbackslash] Start of a command
        \begin{description}
            \item[\command{LaTeX}] \LaTeX
        \end{description}
        \item[\brackets{}] Used for options on commands
        \item[\{\}] Used for arguments on commands
    \end{description}
\end{frame}




\subsection{\braceCommand{documentclass}}

\begin{frame}
\frametitle{\braceCommand{documentclass}}
    Used to define the type of document\\
    Example types of documentclass:
    \begin{description}
        \item[article] Good for scientific journals, presentations, short reports, documentation, \ldots
        \item[report] Good for long reports, small books, thesis, \ldots
        \item[book] Good for writing real books
        \item[slides] Good for simple slides
        \item[beamer] Good for complex presentations
    \end{description}
\end{frame}


\begin{frame}
\frametitle{\braceCommand{documentclass}}
    Options:
    \begin{description}
        \item[pt] Font size. Default is 10pt. Ex: 12pt, 14pt, \ldots
        \item[letterpaper] Size of the paper. letterpaper is default.
        \item[twocolumn] Define if the document should be one column or two
    \end{description}
\end{frame}




\subsection{\braceCommand{usepackage}}

\begin{frame}
\frametitle{\braceCommand{usepackage}}
    Used to add modules/packages to the \LaTeX{} document.\\
    Example packages:
    \begin{description}
        \item[geometry] Used to change geometry including margin and layout
        \item[graphicx] Used to include graphics in the document
        \item[float] Allows changing the position of a graphic in the document/page
    \end{description}
\end{frame}



\subsection{\braceCommand{begin} \&\& \braceCommand{end}}

\begin{frame}
\frametitle{\braceCommand{begin} \&\& \braceCommand{end}}
    Used to start and end \textit{environments} in \LaTeX .\\
    Some environments that use \braceCommand{begin} and 
    \braceCommand{end}:
    \begin{itemize}
        \item title
        \item document
        \item lists
%        \item figure
    \end{itemize}
\end{frame}



\subsection{Lists}

\begin{frame}
\frametitle{Lists}
    \begin{itemize}
        \item Lists come in three flavors:
        \begin{itemize}
            \item itemize
            \item enumerate
            \item description
        \end{itemize}
    \end{itemize}
\end{frame}



\subsubsection{itemize}

\begin{frame}[fragile]
\frametitle{itemize}
\begin{teX}
\begin{itemize}
    \item This is a sample list
    \item to show the code for an itemized list
    \begin{itemize}
        \item sublist
    \end{itemize}
    \item Another item
\end{itemize}
\end{teX}
\end{frame}

\begin{frame}
    \begin{itemize}
        \item This is a sample list
        \item to show the code for an itemized list
        \begin{itemize}
            \item sublist
        \end{itemize}
        \item Another item
   \end{itemize}
\end{frame}

\subsubsection{enumerate}

\begin{frame}[fragile]
\frametitle{enumerate}
\begin{teX}
\begin{enumerate}
    \item This is a sample list
    \item to show the code for an enumerated list
    \begin{enumerate}
        \item sublist
    \end{enumerate}
    \item Another item
\end{enumerate}
\end{teX}
\end{frame}

\begin{frame}
\frametitle{enumerate}
    \begin{enumerate}
        \item This is a sample list
        \item to show the code for an enumerated list
        \begin{enumerate}
            \item sublist
        \end{enumerate}
        \item Another item
   \end{enumerate}
\end{frame}

\subsubsection{description}

\begin{frame}[fragile]
\frametitle{description}
\begin{teX}
\begin{description}
    \item[description] Good for definitions
    \item[something] This shows a second definition
\end{description}
\end{teX}
\end{frame}

\begin{frame}
\frametitle{description}
\begin{description}
    \item[description] Good for definitions
    \item[something] This shows a second definition
\end{description}
\end{frame}


\section{Sectioning}

\begin{frame}
\frametitle{Sectioning}
\begin{description}
    \item[\braceCommand{chapter}] Level 0\\New page\\Only 
    in books and reports
    \item[\braceCommand{section}] Level 1\\Same page
    \item[\braceCommand{subsection}] Level 2\\Same page
    \item[\braceCommand{subsubsection}] Level 3\\Same page
\end{description}
\end{frame}


\section{Math}

\subsection{Math}

\begin{frame}
\frametitle{Math}
\begin{itemize}
    \item Math in \LaTeX{} is beautiful.
    \item Uses the asamath or mathtools package
    \item Can be sectioned off or inline
    \item Relation Symbols, Binary Operations, Greek Letters, Set/Logic
        Notation, Delimiters, Trig Symbols, etc.
    \item Simple example: \$2+2=5\$ -> $2+2=5$
\end{itemize}
\end{frame}

\subsection{Ex.}

\begin{frame}
\frametitle{Math Ex.}
\begin{math}
A_{m,n} =
 \begin{pmatrix}
  a_{1,1} & a_{1,2} & \cdots & a_{1,n} \\
  a_{2,1} & a_{2,2} & \cdots & a_{2,n} \\
  \vdots  & \vdots  & \ddots & \vdots  \\
  a_{m,1} & a_{m,2} & \cdots & a_{m,n}
 \end{pmatrix}
 \end{math}
\end{frame}


\section{Document Ex.}

\begin{frame}
\frametitle{Document Ex.}
\begin{figure}[DocEx] 
  \begin{centering}
  \includegraphics[bb= 111 89 500 621, crop=true, 
                 height=1.3\textheight]{docEx.pdf}
  \caption{Example Document}
  \end{centering}
\end{figure}
\end{frame}


\section{Programs}

\subsection{Latex Packages}

\begin{frame}
\frametitle{\LaTeX{} Packages}
\begin{itemize}
    \item Linux
    \begin{itemize}
        \item TeX Live
        \item TeTeX (For Fedora and RHEL/CentOS)
    \end{itemize}
    \item Mac OS X
    \begin{itemize}
        \item MacTeX
    \end{itemize}
    \item Windows
    \begin{itemize}
        \item MikTeX
        \item ProTeXt
    \end{itemize}
\end{itemize}
\end{frame}

\subsection{IDEs}

\begin{frame}
\frametitle{\LaTeX{} IDEs}
\begin{itemize}
    \item TeXmaker
    \item TeXworks
    \item Vim/LaTeX-suite
    \item Emacs AUCTeX
\end{itemize}
\end{frame}

\subsection{How to Compile}

\begin{frame}
\frametitle{How to Compile}
\begin{itemize}
    \item \texttt{pdflatex}
    \begin{itemize}
        \item \texttt{pdflatex index.tex}
    \end{itemize}
    \item \texttt{latexmk}
    \begin{itemize}
        \item latexmk -pdf index.tex
        \item latexmk -c
    \end{itemize}
\end{itemize}
\end{frame}


\section{Conclusion}
\subsection{References}
\begin{frame}
\frametitle{References}
\begin{description}
    \item[Art of Problem Solving] \url{http://www.artofproblemsolving.com/Wiki/index.php/LaTeX}
    \item[NASA GISS] \url{http://www.giss.nasa.gov/tools/latex/ltx-2.html}
    \item[WikiBooks] \url{http://en.wikibooks.org/wiki/LaTeX}
    \item[Stack Overflow] \url{http://stackoverflow.com}
    \item[This Presentation] \url{https://github.com/rzitex/LaTeX\_Class}
\end{description}
\end{frame}

\subsection{Questions}
\begin{frame}
\begin{beamercolorbox}[center,shadow=true,rounded=true,]{note}
        Questions?
\end{beamercolorbox}
\end{frame} 


\end{document}

